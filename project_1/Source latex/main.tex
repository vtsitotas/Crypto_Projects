\documentclass{article}
\newcommand{\omicron}{o}
\usepackage[utf8]{inputenc}
\usepackage{amsmath, amssymb}
\usepackage{alphabeta} % Για ελληνικούς χαρακτήρες
\usepackage{graphicx}
\usepackage[greek]{babel}
\usepackage{graphicx}

\begin{document}


\section*{Άσκηση 1 (2.2)}

\subsection*{Το κρυπτογραφημένο μήνυμα:}
"οκηθμφδζθγοθχυκχσφθμφμχγ"

\subsection*{Μετατροπή σε αριθμητική μορφή:}
$15, 10, 7, 8, 12, 21, 4, 6, 8, 3, 15, 8, 22, 20, 10, 22, 18, 21, 8, 12, 21, 12, 22, 3$

\subsection*{Εύρεση της ρίζας $x_0$ του τριωνύμου}
Δίνεται το πολυώνυμο:
\[ g(x) = x^2 + 3x + 1 \]

Η ρίζα του τριωνύμου βρίσκεται λύνοντας:
\[ x^2 + 3x + 1 = 0 \]

Χρησιμοποιούμε Διακρίνουσα:
\[
 x = \frac{-b \pm \sqrt{b^2 - 4ac}}{2a}
\]
Όπου:
$a = 1$, $b = 3$, $c = 1$

Υπολογίζουμε τη διακρίνουσα:
\[ \Delta = 3^2 - 4(1)(1) = 9 - 4 = 5 \]

Οι ρίζες είναι:
\[ x_0 = \frac{-3 \pm \sqrt{5}}{2} \]
Παίρνουμε τη θετική ρίζα:
\[ x_0 = \frac{-3 + \sqrt{5}}{2} \]

\subsection*{Υπολογισμός του $f(x_0)$}
Δίνεται το πολυώνυμο:
\[ f(x) = x^5 + 3x^3 + 7x^2 + 3x^4 + 5x + 4 \]
Αντικαθιστούμε $x = x_0$ και χρησιμοποιούμε τη σχέση:
\[ x_0^2 = -3x_0 -1 \]
για να απλοποιήσουμε τις εκφράσεις.

Με τους υπολογισμούς καταλήγουμε:
\[ f(x_0) = 3 \]

\subsection*{Αφαίρεση του $f(x_0) = 3$ από κάθε αριθμό}
Αφαιρούμε 3 από κάθε αριθμό:
\[ (15-3, 10-3, 7-3, 8-3, 12-3, 21-3, 4-3, 6-3, 8-3, 3-3, \dots) \]

Σημείωση: Το 0 δεν αντιστοιχεί σε γράμμα, πιθανό σφάλμα στην κρυπτογράφηση.

\subsection*{Βήμα 3: Αντιστοίχιση αριθμών σε γράμματα}
Χρησιμοποιούμε τον πίνακα αντιστοίχισης:

12 $\rightarrow$ μ

7 $\rightarrow$ η

4 $\rightarrow$ δ

5 $\rightarrow$ ε

9 $\rightarrow$ ι

18 $\rightarrow$ σ

1 $\rightarrow$ α

3 $\rightarrow$ γ

5 $\rightarrow$ ε

(0 δεν υπάρχει, πιθανό λάθος κρυπτογράφησης και θα έπρεπε να ειναι ω)

12 $\rightarrow$ μ

5 $\rightarrow$ ε

19 $\rightarrow$ τ

17 $\rightarrow$ ρ

7 $\rightarrow$ η

19 $\rightarrow$ τ

15 $\rightarrow$ ο

18 $\rightarrow$ σ

5 $\rightarrow$ ε

9 $\rightarrow$ ι

18 $\rightarrow$ σ

9 $\rightarrow$ ι

19 $\rightarrow$ τ

(0 δεν υπάρχει, πιθανό λάθος κρυπτογράφησης και θα έπρεπε να ειναι ω)

Άρα το αποκρυπτογραφημένο μήνυμα είναι:
"μηδεισαγεωμετρητοςεισιτω"

Προφανώς, το σωστό μήνυμα είναι:\textbf{"Μηδείς αγεωμέτρητος εισίτω"}
,
μια φράση που ήταν γραμμένη στην είσοδο της Ακαδημίας του Πλάτωνα.

\subsection*{Συμπέρασμα}
Το σφάλμα στα "0" μπορεί να οφείλεται σε λάθος αρχική κρυπτογράφηση ή απλά σε μια μετατόπιση που δεν διατηρεί το αλφάβητο μέσα στα έγκυρα όρια (1-24). Ωστόσο, η λογική της αποκρυπτογράφησης ήταν σωστή και το τελικό μήνυμα επιβεβαιώνει την επιτυχή ανάκτηση του αρχικού κειμένου.

\section*{Άσκηση 2 (2.3)}
Αυτή η μέθοδος χρησιμοποιεί:

\subsection*{1. Δοκιμή $Friedman$}
Υπολογίζει τον \textbf{δείκτη σύμπτωσης (IC)} για διαφορετικά μήκη κλειδιού $\rho$:
\[ IC = \frac{\sum_{i=1}^{n} m_i(m_i-1)}{k(k-1)} \]
όπου $m_i$ είναι οι εμφανίσεις του $i$-οστού γράμματος και $k$ το μήκος κειμένου. Όταν ο IC πλησιάζει την τιμή \textbf{0.0665} (τυπική για αγγλικά κείμενα), έχουμε πιθανό μήκος κλειδιού.

\subsection*{2. Ανάλυση Συχνότητας}
Για κάθε ομάδα χαρακτήρων:
\begin{itemize}
\item Μετράμε τις συχνότητες εμφάνισης
\item Δοκιμάζουμε όλες τις πιθανές μετατοπίσεις (0-25)
\item Επιλέγουμε τη μετατόπιση που ελαχιστοποιεί την απόσταση από τις γνωστές αγγλικές συχνότητες
\end{itemize}

\subsection*{3. Αποκρυπτογράφηση}
Εφαρμόζουμε το κλειδί $K = (k_1,k_2,...,k_\rho)$ για να αναιρέσουμε τη μετατόπιση:
\[ P_i = (C_i - k_{i \mod \rho}) \mod 26 \]

\section*{Άσκηση 3 (2.4)}

\subsection*{Λογική Επίλυσης}

Έστω ότι δίνεται η σχέση κωδικοποίησης:
\[
c = m \oplus (m \ll 6) \oplus (m \ll 10)
\]
όπου \( \ll \) δηλώνει κυκλική μετατόπιση προς τα αριστερά και \( \oplus \) η πράξη $XOR$ 

Θέλουμε να βρούμε τον τύπο για την αποκωδικοποίηση, δηλαδή να εκφράσουμε το \( m \) ως συνάρτηση του \( c \). Λόγω των ιδιοτήτων της $XOR$ και της γραμμικότητάς της στο δυαδικό πεδίο $(GF(2))$, μπορούμε να επαναδιατυπώσουμε τη σχέση ως:

\[
m = c \oplus (m \ll 6) \oplus (m \ll 10)
\]

Αυτή η εξίσωση επιλύεται επαναληπτικά, ξεκινώντας με αρχική τιμή \( m = 0 \), και εφαρμόζοντας τον παραπάνω τύπο έως ότου το αποτέλεσμα σταθεροποιηθεί.

\subsection*{Περιγραφή Κώδικα $Python$}

Ο κώδικας αποτελείται από τις παρακάτω βασικές συνιστώσες:

\begin{itemize}
  \item \textbf{$cyclic\_left\_shift(x, shift, bits)$}: Υλοποιεί κυκλική μετατόπιση \textit{αριστερά} \( \ll \), για αριθμούς 16-bit.
  \item \textbf{$encode(m)$}: Υπολογίζει το \( c \) με βάση τον τύπο κωδικοποίησης.
  \item \textbf{$decode(c)$}: Ξεκινά με \( m = 0 \) και επαναλαμβάνει τον τύπο αποκωδικοποίησης για 16 επαναλήψεις, έως ότου υπολογιστεί το αρχικό μήνυμα \( m \).
  \item \textbf{Επαλήθευση}: Παράγεται τυχαίο μήνυμα \( m \), κωδικοποιείται σε \( c \), και η αποκωδικοποίηση επαληθεύει την ορθότητα της μεθόδου με σύγκριση των τιμών \( m \) και \( m' \).
\end{itemize}

Η μέθοδος είναι αποτελεσματική και συγκλίνει λόγω της γραμμικής φύσης της εξίσωσης και του περιορισμένου εύρους (16-bit) των αριθμών.

\section*{Άσκηση 4 (2.6)}
\subsection*{Περιγραφή Κώδικα $One Time Pad$}

Ο παρών αλγόριθμος υλοποιεί την κρυπτογράφηση και αποκρυπτογράφηση κειμένου με τη μέθοδο \textbf{$One Time Pad (OTP)$} χρησιμοποιώντας προκαθορισμένη 5-bit δυαδική αναπαράσταση χαρακτήρων.

\begin{itemize}
  \item Ορίζεται ένας πίνακας αντιστοίχισης χαρακτήρων σε $5-bit$ δυαδικά $strings$ και αντίστροφα.
  \item Το μήνυμα μετατρέπεται σε μία ακολουθία $bits$ με βάση τον πίνακα κωδικοποίησης.
  \item Δημιουργείται ένα τυχαίο δυαδικό κλειδί ίδιου μήκους με το μήνυμα.
  \item Η κρυπτογράφηση υλοποιείται με $bitwise$ τελεστή $XOR$ ανάμεσα στο μήνυμα και το κλειδί.
  \item Το αποτέλεσμα της κρυπτογράφησης μετατρέπεται ξανά σε χαρακτήρες.
  \item Η αποκρυπτογράφηση υλοποιείται εφαρμόζοντας το $XOR$ μεταξύ του κρυπτογραφημένου μηνύματος και του ίδιου κλειδιού, επιστρέφοντας το αρχικό μήνυμα.
\end{itemize}

Ο κώδικας περιλαμβάνει συναρτήσεις για:
\begin{enumerate}
  \item Μετατροπή χαρακτήρων σε bits και αντίστροφα.
  \item Δημιουργία τυχαίου κλειδιού.
  \item Κρυπτογράφηση και αποκρυπτογράφηση με χρήση $XOR$.
\end{enumerate}

Ο χρήστης εισάγει ένα μήνυμα και λαμβάνει ως έξοδο το κρυπτογραφημένο κείμενο, το κλειδί, καθώς και το αποκρυπτογραφημένο μήνυμα για επιβεβαίωση της ορθότητας.
\section*{Άσκηση 5 (3.3)}
Χρησιμοποιήσαμε τη γεννήτρια \textbf {$LCG (Linear Congruential Generator)$} με τις παρακάτω παραμέτρους:
\begin{itemize}
    \item $m = 2^{10} = 1024$ (τροποποιητής)
    \item $a = 5$ (πολλαπλασιαστής)
    \item $c = 3$ (σταθερά)
    \item $x_0 = 1$ (αρχική τιμή)
\end{itemize}

Η αναδρομική συνάρτηση που παράγει την ακολουθία είναι:
\[
x_{i+1} = (a \cdot x_i + c) \mod m
\]

Υπολογίστηκαν οι πρώτοι $150$ όροι της ακολουθίας και αναλύθηκαν οι συχνότητες εμφάνισης των τιμών.

\subsection*{Ιστόγραμμα Συχνοτήτων}

Το διάστημα τιμών $[0, 1023]$ χωρίστηκε σε $10$ ισομεγέθη διαστήματα ($bins$), και για κάθε ένα μετρήθηκε πόσες φορές εμφανίστηκαν οι τιμές της ακολουθίας εντός αυτού. Το παρακάτω ιστόγραμμα δείχνει τις συχνότητες εμφάνισης:

\begin{center}
    \includegraphics[width=0.8\textwidth]{histogram.png}
\end{center}

\subsection*{Παρατηρήσεις και Συμπεράσματα}

Η κατανομή των τιμών \textbf{δεν είναι απολύτως ομοιόμορφη}. Παρατηρήθηκαν διακυμάνσεις στις συχνότητες μεταξύ των διαστημάτων, ενώ ο αριθμός των διαφορετικών τιμών της ακολουθίας ήταν περιορισμένος σε σχέση με το πλήθος των δυνατών τιμών ($1024$). Αυτό οφείλεται στο ότι οι παράμετροι της $LCG$ δεν εξασφαλίζουν μέγιστο μήκος κύκλου.

\begin{itemize}
    \item Η ακολουθία δεν κάλυψε όλες τις τιμές στο διάστημα $[0, 1023]$.
    \item Ορισμένα διαστήματα είχαν εμφανώς περισσότερες τιμές.
    \item Η κατανομή αποκλίνει από την ιδανική ομοιόμορφη κατανομή.
\end{itemize}

Για να προσεγγιστεί καλύτερα η ομοιόμορφη κατανομή, προτείνεται:
\begin{itemize}
    \item Χρήση παραμέτρων που εξασφαλίζουν \textbf{μέγιστο κύκλο} (π.χ. κατάλληλο $a$, $c$).
    \item Αύξηση του αριθμού παραγόμενων τιμών (π.χ. 1000 ή περισσότερες).
\end{itemize}
\section*{Άσκηση 6 (3.8)}
    \begin{itemize}
    \item \textlatin{list\_to\_string(l)}: Ενώνει τα στοιχεία μιας λίστας σε ένα ενιαίο αλφαριθμητικό.
    
    \item \textlatin{string\_xor(btext, key)}: Εκτελεί πράξη $XOR$ μεταξύ δύο δυαδικών αλφαριθμητικών ίδιου μήκους.
    
    \item \textlatin{text\_enc(text)}: Κωδικοποιεί ένα κείμενο σε δυαδική μορφή $5-bit$ ανά χαρακτήρα, σύμφωνα με το λεξικό \textlatin{aDict}.
    
    \item \textlatin{text\_dec(binary\_string)}: Αποκωδικοποιεί μια δυαδική συμβολοσειρά σε κανονικό κείμενο, αντιστρέφοντας τη χαρτογράφηση του \textlatin{aDict}.
    
    \item \textlatin{sumxor(l)}: Υπολογίζει το $XOR$ όλων των στοιχείων μιας λίστας δυαδικών τιμών.
    
    \item \textlatin{lfsr(seed, feedback, bits)}: Υλοποιεί καταχωρητή ολίσθησης με γραμμική ανάδραση ($LFSR$), με αρχικό \textlatin{seed}, θέσεις ανάδρασης \textlatin{feedback} και πλήθος παραγόμενων $bits$ 
    
    \item \textlatin{find\_seed(plaintext, ciphertext, feedback, aDict)}: Προσπαθεί να βρει τον αρχικό \textlatin{seed} του $LFSR$, συγκρίνοντας το αναμενόμενο \textlatin{keystream} που προκύπτει από το $XOR$ του \textlatin{plaintext} και \textlatin{ciphertext}, με παραγόμενα \textlatin{keystreams}.
    
    \item \textlatin{main()}: Κεντρική $main$ που εκτελεί την κωδικοποίηση του μηνύματος, βρίσκει τον σωστό \textlatin{seed}, παράγει το \textlatin{keystream} και αποκωδικοποιεί το κρυπτογραφημένο μήνυμα.
\end{itemize}
\section*{Άσκηση 7 (3.11)}
\subsection*{Περιγραφή Κώδικα}

Ο κώδικας υλοποιεί την κρυπτογράφηση και αποκρυπτογράφηση ενός μηνύματος χρησιμοποιώντας τον αλγόριθμο $RC4$ και μία $5-bit$ προσαρμοσμένη δυαδική αναπαράσταση χαρακτήρων.

\subsection*{1. Μετατροπές Χαρακτήρων}
\begin{itemize}
    \item \texttt{$aDict$}: Λεξικό που αντιστοιχεί χαρακτήρες σε 5-$bit$ δυαδικό 
    \item \texttt{$reverseDict$}: Αντιστροφή του \texttt{$aDict$}, μετατρέπει 5-bit δυαδικά σε χαρακτήρες.
\end{itemize}

\subsection*{2. $RC4$ $Key$ $Scheduling$ $Algorithm$ ($KSA$)}
Η συνάρτηση \texttt{$ksa(key)$} αρχικοποιεί τον πίνακα $S[0..255]$ με βάση το κλειδί (\texttt{$HOUSE$}) και τον ανακατεύει χρησιμοποιώντας αλγοριθμική συνάρτηση.

\subsection*{3. $RC4$ $Pseudo-Random$ $Generation$ $Algorithm$ ($PRGA$)}
Η συνάρτηση \texttt{$prga(S, n)$} παράγει μια ακολουθία \texttt{n} ψευδοτυχαίων $bytes$ ($keystream$), που χρησιμοποιείται για το $XOR$ με το μήνυμα.

\subsection*{4. Κωδικοποίηση και Αποκωδικοποίηση}
\begin{itemize}
    \item \texttt{$text\_to\_binary(text)$}: Μετατρέπει το μήνυμα σε συνεχόμενο δυαδικό $string$ χρησιμοποιώντας το \texttt{$aDict$}.
    \item \texttt{$keystream\_to\_binary(keystream, length)$}: Μετατρέπει κάθε $byte$ του $keystream$ σε 8-$bit$ δυαδικό $string$ και το περικόπτει στο κατάλληλο μήκος.
    \item \texttt{$xor(bin\_str, keystream)$}: Υπολογίζει $bitwise$ $XOR$ ανάμεσα στο δυαδικό μήνυμα και το $keystream$.
    \item \texttt{$binary\_to\_text(binary)$}: Μετατρέπει το αποκρυπτογραφημένο δυαδικό πίσω σε χαρακτήρες μέσω του \texttt{$reverseDict$}.
\end{itemize}

\subsection*{5. Ροή Προγράμματος}
\begin{enumerate}
    \item Το αρχικό κείμενο μετατρέπεται σε δυαδικό string 5-$bit$ ανά χαρακτήρα.
    \item Δημιουργείται το $keystream$ από το κλειδί με χρήση του $RC4$.
    \item Πραγματοποιείται $XOR$ για την κρυπτογράφηση.
    \item Επαναλαμβάνεται το $XOR$ για αποκρυπτογράφηση.
    \item Το αποτέλεσμα μετατρέπεται πίσω σε αναγνώσιμο κείμενο.
\end{enumerate}

\section*{Άσκηση 8 (4.3)}

Η διαφορική ομοιομορφία ενός $S-box$, δηλαδή το ${Diff}(S)$, είναι ένα από τα πιο σημαντικά χαρακτηριστικά αντοχής του σε επιθέσεις διαφορικής κρυπτανάλυσης. Ορίζεται ως:

\[
\text{$Diff$}(S) = \max_{\substack{x \in \{0,1\}^n \setminus \{0\} \\ y \in \{0,1\}^m}} \left| \left\{ z \in \{0,1\}^n : S(z \oplus x) \oplus S(z) = y \right\} \right|
\]

Δηλαδή, πρόκειται για το μέγιστο πλήθος λύσεων $z$ για τις οποίες η διαφορική εξίσωση $S(z \oplus x) \oplus S(z) = y$ ισχύει, για κάθε μη μηδενικό $x$ και για κάθε $y$.

Μια γενική κατώτερη εκτίμηση για τη διαφορική ομοιομορφία είναι:

\[
\text{$Diff$}(S) \geq \max \left\{ 2, 2^{n-m} \right\}
\]

Για ένα $S$-$box$ που μετασχηματίζει από $n=6$ $bit$ σε $m=4$ $bit$, δηλαδή $S: \{0,1\}^6 \to \{0,1\}^4$, έχουμε:

\[
\text{$Diff$}(S) \geq \max \left\{ 2, 2^{6-4} \right\} = \max \left\{ 2, 4 \right\} = 4
\]

Αυτό σημαίνει πως οποιοδήποτε $S$-$box$ που υλοποιεί τέτοιο μετασχηματισμό, θεωρητικά δεν μπορεί να έχει διαφορική ομοιομορφία μικρότερη του 4. Όσο μικρότερη είναι η τιμή του $\text{$Diff$}(S)$, τόσο πιο ανθεκτικό είναι το $S$-$box$ στις διαφορικές επιθέσεις.

\bigskip

Σύμφωνα με την εκτέλεση του $script$ που υλοποιεί τον παραπάνω υπολογισμό για το συγκεκριμένο $S$-$box$ που δίνεται, βρέθηκε ότι:

\[
\text{$Diff$}(S) = 14
\]

Αυτό σημαίνει ότι για ορισμένες τιμές διαφορών εισόδου $x$ και διαφορών εξόδου $y$, υπάρχουν έως και 14 διαφορετικές εισόδους $z$ για τις οποίες η διαφορική εξίσωση ισχύει. Η τιμή αυτή είναι σημαντικά μεγαλύτερη από το ελάχιστο θεωρητικά επιτρεπτό όριο (4), και συνεπώς το συγκεκριμένο $S$-$box$ είναι ευάλωτο σε διαφορικές επιθέσεις και δεν ενδείκνυται για χρήση σε σύγχρονα κρυπτοσυστήματα που απαιτούν ισχυρή αντίσταση.

\bigskip

Ο πίνακας κατανομής που προκύπτει από το $script$ δείχνει την πλήρη κατανομή του πλήθους λύσεων της εξίσωσης για κάθε $(x,y)$ και επιβεβαιώνει τη μέγιστη τιμή $14$.
\bigskip
\bigskip


\section*{Άσκηση 9 (4.4)}
\subsection*{Υπολογισμός του $Linear$ $Branch$ $Number$ ($LBN$)}

Ορίζουμε το $Linear Branch Number (LBN)$ ενός $S$-κιβωτίου $S: \{0,1\}^m \to \{0,1\}^m$ ως:

\[
\text{$LBN$}(S) = \min_{\alpha, \beta \in \mathbb{F}_2^m \setminus \{0\}, \, C_S(\alpha, \beta) \ne 0} \left( \text{$wt$}(\alpha) + \text{$wt$}(\beta) \right)
\]

όπου:
\begin{itemize}
    \item $ \text{$wt$}(\cdot) $ είναι το βάρος $Hamming$ (πόσα $bits$ είναι 1),
    \item $ C_S(\alpha, \beta) = \sum_{x \in \mathbb{F}_2^m} (-1)^{\beta \cdot S(x) + \alpha \cdot x} $ είναι ο συντελεστής συσχέτισης.
\end{itemize}

Ο στόχος είναι να υπολογίσουμε το $LBN$ του $S$-κιβωτίου που ορίζεται από τον τύπο:

\[
S(x) = (x^2 + 3) \mod 32
\]

όπου $x$ είναι η δεκαδική αναπαράσταση του δυαδικού $x$.

 1. Ορισμός του $LBN$

Το $LBN$ υπολογίζεται ως το ελάχιστο άθροισμα των βαρών $Hamming$ των διανυσμάτων $\alpha$ και $\beta$ που οδηγούν σε μη μηδενική τιμή για τον συντελεστή συσχέτισης:

\[
\text{$LBN$}(S) = \min_{\alpha, \beta \in \mathbb{F}_2^m \setminus \{0\}, \, C_S(\alpha, \beta) \ne 0} \left( \text{$wt$}(\alpha) + \text{$wt$}(\beta) \right)
\]

Για να έχει $ \text{$LBN$}(S) = 2 $, πρέπει να βρούμε ζεύγη $(\alpha, \beta)$ που να ικανοποιούν:

\[
\text{$wt$}(\alpha) + \text{$wt$}(\beta) = 2
\]

και για τα οποία $ C_S(\alpha, \beta) \neq 0$.

 2. Πόσα τέτοια ζεύγη υπάρχουν;

Για κάθε διανύσμα $\alpha \in \mathbb{F}_2^5$ με βάρος $1$, έχουμε:

\[
\binom{5}{1} = 5 \quad \text{επιλογές για} \, \alpha
\]

Το ίδιο ισχύει και για $\beta$, οπότε υπάρχουν συνολικά:

\[
5 \times 5 = 25 \quad \text{πιθανοί συνδυασμοί} \, (\alpha, \beta)
\]

με βάρος 2, δηλαδή:

\[
\text{$wt$}(\alpha) + \text{$wt$}(\beta) = 2
\]

 3. Εξέταση της τιμής του $C_S(\alpha, \beta)$

Για κάθε ζεύγος $(\alpha, \beta)$, υπολογίζουμε τον συντελεστή συσχέτισης:

\[
C_S(\alpha, \beta) = \sum_{x = 0}^{31} (-1)^{\beta \cdot S(x) + \alpha \cdot x}
\]

Από τα 25 ζεύγη, μόνο **8** έχουν μη μηδενική τιμή για τον συντελεστή συσχέτισης $C_S(\alpha, \beta) \neq 0$.

 4. Συμπέρασμα

Η ελάχιστη τιμή που μπορεί να έχει το $LBN$ είναι 2, και αυτή επιτυγχάνεται μόνο για 8 ζεύγη $(\alpha, \beta)$. Επομένως, το $Linear Branch Number$ του $S$-κιβωτίου είναι:

\[
\text{$LBN$}(S) = 2
\]

Αυτό δείχνει ότι το $S$-κιβώτιο προσφέρει μια μέτρια σύγχυση ($confusion$), καθώς το $LBN$ δεν είναι κοντά στην μέγιστη τιμή που είναι 6 (δηλαδή $m + 1$).


\section*{Άσκηση 10 (4.8)}

\subsection{Ανάλυση του $Avalanche$ $Effect$ στον $AES-128$}

To $avalanche$ $effect$ είναι μία βασική ιδιότητα που πρέπει να πληρεί κάθε ασφαλής αλγόριθμος κρυπτογράφησης. Στην περίπτωση του $AES-128$, το $avalanche$ $effect$ αναφέρεται στη συμπεριφορά του αλγορίθμου όταν το αρχικό μήνυμα (ή $plaintext$) υποβάλλεται σε μία μικρή τροποποίηση (π.χ., η αλλαγή ενός μόνο $bit$). Η ιδέα είναι ότι μία τόσο μικρή αλλαγή στο αρχικό μήνυμα θα πρέπει να προκαλέσει σημαντική αλλαγή στο κρυπτογραφημένο κείμενο ($ciphertext$), με το αποτέλεσμα να είναι ότι σχεδόν όλα τα $bits$ του κρυπτογραφημένου κειμένου να αλλάξουν.

\subsection{Δημιουργία Ζευγαριών Μηνυμάτων}

Για την ανάλυση του $avalanche$ $effect$, δημιουργήσαμε πάνω από 30 ζευγάρια μηνυμάτων \( m_1 \) και \( m_2 \), όπου τα μηνύματα διαφέρουν σε ακριβώς 1 $bit$. Συγκεκριμένα:
\begin{itemize}
    \item Τα μηνύματα \( m_1 \) και \( m_2 \) έχουν μήκος 256 $bits$ (32 $bytes$), το οποίο είναι διπλάσιο από το μήκος ενός $block$ του $AES-128$ ($128$ $bits$ ή 16 $bytes$).
    \item Το μήνυμα \( m_2 \) προκύπτει από το μήνυμα \( m_1 \) αλλάζοντας μόνο ένα $bit$.
\end{itemize}

\subsection{Κρυπτογράφηση με $AES-128$}

Η κρυπτογράφηση των μηνυμάτων έγινε χρησιμοποιώντας τον αλγόριθμο $AES-128$ και δύο καταστάσεις λειτουργίας:
\begin{itemize}
    \item $ECB$ ($Electronic$ $Codebook$): Κρυπτογράφηση με ανεξάρτητη κρυπτογράφηση κάθε μπλοκ.
    \item $CBC$ ($Cipher$ $Block$ $Chaining$): Κρυπτογράφηση με αλυσίδωση των μπλοκ και χρήση του $IV$ ($Initialization$ $Vector$).
\end{itemize}

Για κάθε ζευγάρι μηνυμάτων \( (m_1, m_2) \), κρυπτογραφήθηκαν τα μηνύματα και στη συνέχεια υπολογίστηκε η διαφορά σε $bits$ μεταξύ των αντίστοιχων κρυπτογραφημένων μηνυμάτων.

\subsection{Υπολογισμός Διαφοράς σε $Bits$}

Η διαφορά σε $bits$ μεταξύ δύο κρυπτογραφημένων μηνυμάτων υπολογίστηκε χρησιμοποιώντας την \textbf{$Hamming$ $distance$}, η οποία μετρά πόσα $bits$ διαφέρουν μεταξύ δύο κωδικοποιημένων δεδομένων. Στην περίπτωση αυτή, η διαφορά σε $bits$ για κάθε ζευγάρι μηνυμάτων υπολογίστηκε για τις λειτουργίες $ECB$ και $CBC$.

\subsection{Αποτελέσματα και Ανάλυση}

Τα αποτελέσματα του τεστ έδειξαν ότι:
\begin{itemize}
    \item Στην $ECB$, αν και τα μπλοκ κρυπτογραφούνται ανεξάρτητα το ένα από το άλλο, η διαφορά στα κρυπτογραφημένα μηνύματα για κάθε ζευγάρι μηνυμάτων ήταν σχετικά υψηλή, αν και ίσως όχι τόσο έντονη όσο στην $CBC$.
    \item Στην $CBC$, η διαφορά στα κρυπτογραφημένα μηνύματα ήταν πιο έντονη λόγω της αλυσίδωσης των μπλοκ και της χρήσης του IV. Η αλλαγή ενός $bit$ στο αρχικό μήνυμα προκαλεί μεγάλες αλλαγές όχι μόνο στο αντίστοιχο μπλοκ, αλλά και στο επόμενο μπλοκ.
\end{itemize}

Η μέση διαφορά σε $bits$ που παρατηρήθηκε ήταν κοντά στο 50\% του μήκους του κρυπτογραφημένου μηνύματος (128 $bits$), το οποίο επιβεβαιώνει την ύπαρξη του avalanche effect στον AES-128.

\subsection{Συμπέρασμα}

Από τα αποτελέσματα της ανάλυσης, μπορούμε να συμπεράνουμε ότι ο $AES$-128 εκπληρώνει την απαίτηση για το $avalanche$ $effect$, με την αλλαγή ενός μόνο $bit$ στο αρχικό μήνυμα να προκαλεί σημαντική αλλαγή στο κρυπτογραφημένο κείμενο. Η λειτουργία $CBC$ προσφέρει ισχυρότερο $avalanche$ $effect$ σε σχέση με την $ECB$, καθώς η αλυσίδωση των μπλοκ και η χρήση του $IV$ ενισχύουν την ασφάλεια του αλγορίθμου.
\subsection{Ανάλυση Κώδικα}

\begin{itemize}
    \item \textbf{Κύριες Συναρτήσεις}:
    \begin{itemize}
        \item \texttt{$hamming$\_$distance$($a$, $b$)}: Υπολογίζει την απόσταση $Hamming$ μεταξύ δύο $bytestrings$.
        \item \texttt{$main$\_$execution$\_$block$}: Δημιουργεί μηνύματα, εφαρμόζει $bit-flip$, και εκτελεί κρυπτογραφήσεις.
    \end{itemize}
    
    \item \textbf{Βασικές Μεταβλητές}:
    \begin{itemize}
        \item \texttt{$key$}: 128-$bit$ τυχαίο κλειδί ($AES$-128)
        \item \texttt{$iv$}: 128-$bit$ τυχαίο διάνυσμα αρχικοποίησης ($CBC$)
        \item \texttt{$msg$\_$length$}: 32 $bytes$ (256 $bits$, 2 $blocks$)
    \end{itemize}
\end{itemize}

\subsection{Γενική Λειτουργία}

\begin{enumerate}
    \item Δημιουργία 30 ζευγαριών μηνυμάτων $ (m_1, m_2) $ με διαφορά 1 $bit$.
    \item Κρυπτογράφηση σε δύο λειτουργίες:
    \begin{itemize}
        \item \textbf{$ECB$}: Κάθε $block$ κρυπτογραφείται ανεξάρτητα.
        \item \textbf{$CBC$}: Χρήση $IV$ και αλυσιδωτής σύνδεσης $blocks$.
    \end{itemize}
    \item Υπολογισμός διαφοράς $bits$ στα κρυπτογραφήματα.
\end{enumerate}




\section*{Σημείωση}

Για την ευκολία της συγγραφής και την αισθητική παρουσίαση της αναφοράς, χρησιμοποιήθηκε σε μικρό βαθμό η βοήθεια του \textbf{$ChatGPT$}, κυρίως:
\begin{itemize}
    \item για τη μορφοποίηση του αρχείου \LaTeX\ και το στήσιμο του ιστογράμματος,
    \item για τη βελτίωση κάποιων κομματιών κώδικα, όπως οι περιττοί και χρονοβόροι έλεγχοι τα $labels$ κλπ.
    \item για εύρεση επιπρόσθετων πληροφοριών
\end{itemize}

Η κατανόηση και η επίλυση των ασκήσεων έγιναν από εμένα και η χρήση του εργαλείου έγινε απλώς για διευκόλυνση και όχι για αντικατάσταση της διαδικασίας μάθησης.
\end{document}
